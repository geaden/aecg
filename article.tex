% This is samplepaper.tex, a sample chapter demonstrating the
% LLNCS macro package for Springer Computer Science proceedings;
% Version 2.21 of 2022/01/12
%
\documentclass[runningheads, final]{llncs}
%
\usepackage[T1]{fontenc}
% T1 fonts will be used to generate the final print and online PDFs,
% so please use T1 fonts in your manuscript whenever possible.
% Other font encondings may result in incorrect characters.
%
\usepackage{graphicx}
% Constrain images box
\usepackage[Export]{adjustbox}
% Used for displaying a sample figure. If possible, figure files should
% be included in EPS format.
%
% If you use the hyperref package, please uncomment the following two lines
% to display URLs in blue roman font according to Springer's eBook style:
\usepackage[hidelinks]{hyperref}
\renewcommand\UrlFont{\color{blue}\rmfamily}
\urlstyle{rm}
% Math
% \usepackage{amsthm}
\usepackage{amsmath}
\usepackage{amssymb}
\usepackage[makeroom]{cancel}
% Algorithms
\usepackage[ruled, linesnumbered]{algorithm2e}
\usepackage{algpseudocode}
% Fine tuning of references.
\usepackage[english]{cleveref}
\usepackage[dvipsnames, table]{xcolor}
\usepackage{hyphenat}
\usepackage{cite}
%
% Custom commands
\newcommand*{\eq}[1]
{
  \begin{equation*}
    #1
  \end{equation*}
}
\newcommand{\norm}[1]{\left\lVert#1\right\rVert}
\newcommand{\vprod}[1]{\langle#1\rangle}
\newcommand{\errgrad}{\hat{g}}
\begin{document}
%
\title{Adaptive Frank-Wolfe Method With Relative-Error }
%
%\titlerunning{Abbreviated paper title}
% If the paper title is too long for the running head, you can set
% an abbreviated paper title here
%
\author{Fedor Stonyakin\inst{1}\orcidID{0000-1111-2222-3333} \and
    Mohammad Alkousa\inst{2,3}\orcidID{1111-2222-3333-4444} \and
    Gennady Denisov\inst{3}\orcidID{2222--3333-4444-5555}}
%
\authorrunning{F. Stonyakin, M. Alkousa et al.}
% First names are abbreviated in the running head.
% If there are more than two authors, 'et al.' is used.
%
\institute{Moscow Institute of Physics and Technology
    Moscow, Russia\\
    \email{fedyor@mail.ru} \and
    Innopolis University, Innopolis, Russia\\
    \email{mohammad.math84@gmail.com} \and
    Moscow Institute of Physics and Technology
    Moscow, Russia\\
    \email{denisov.ga@phystech.su}}
%
\maketitle              % typeset the header of the contribution
%
\begin{abstract}
    \textbf{TBD} The abstract should briefly summarize the contents of the paper in
    150--250 words.

    \keywords{Frank-Wolfe method\and Adaptive Optimization method \and Erroneous Conditional Gradient.}
\end{abstract}
%
%
%
\section{Introduction}
\subsection{Problem formulation}

In this paper we study adaptive variance of Frank-Wolfe
\cite{frankwolfe:1956} algorithm with relative-error.

Classical Frank-Wolfe method relies on gradient-oracle to find optimal solution
for optimization problem, which is generally stated as follows:

\begin{equation}\label{eq:optimization}
    \min_{x \in \mathcal{C}} f(x)
\end{equation}

where:

\begin{itemize}
    \item $f$: $\mathbb{R}^n \rightarrow \mathbb{R}$ is continuous
          differentiable
    \item $\mathcal{C} \subseteq \mathbb{R}^n$ is a closed and convex set.
\end{itemize}

We research convergence rate of Adaptive Erroneous Conditional Gradient (AECG)
as proposed by \cite{hallak:2024}.
Unlike classical Frank-Wolf algorithm, its erroneous-conditional algorithm
variance obtains gradient via \textit{Erroneous Oracle (EO)}.

For computational results in this study we use practical problem called
PageRank, a famous method to return fine search results, which was proposed by
Google LLC founders: Sergey Brin and Larry Page \cite{brin:2012}.

According to \cite{anikin:2022}, PageRank optimization problem can be stated
as follows:

\begin{equation}
    f(x) = \frac{1}{2}\norm{Ax}_2^2 \to \min_{x\in\Delta^n_{1}}
\end{equation}

, where $A = I - P^\intercal$, $I$ -- identity matrix of size $n\times n$,
$P$ -- stochastic transition matrix, $P \in \mathbb{R}^{n\times n}$

\textbf{Literature}. While classical conditional gradient methods are well known
and presented in many source \cite{dvurechensky:2015,dvurechensky:2017}, there
are not so many sources on adaptive conditional gradient optimization methods
with relative-errors. In \cite{hallak:2024} the question on conditional
gradients with relative-errors arises. Conditional gradient method,
like Frank-Wolfe \cite{frankwolfe:1956}, have a wide range of practical usage,
especially in web search results page ranking problem (\cite{anikin:2022}).

\textbf{Outline}. Computational experiments for the proposed problem are
described in section \ref{sec:experiments}. We show results for two scenarios:
one is with the step size, which depends on number of iterations, the other one
-- the step size depends on Lipschitz-gradient constant.

In section \ref{sec:discussion} we discuss results of computational experiments,
show how theory and practice may differ.

Section \ref{sec:conclusion} draws a conclusion of the given study and suggests
future work in the field.

\subsection{Mathematical preliminaries}

The standard notation is used throughout the paper. $\norm{\cdot}$ stands for
Euclidean norm. To distinguish between erroneous-gradient and gradient, the
following notation is used, respectively:
$\errgrad \in \mathcal{O}(x, \varepsilon)$ and $g = \nabla f(x)$.

\textit{Linear minimization oracle (LMO)}, which is used in current study
algorithm implementation, is box constrained. Box constraints are popular in
mathematical modeling and widely used.

\begin{definition}[box set]
    A set $\mathcal{C} \subseteq \mathbb{R}^n$ is called a box iff
    $\mathcal{C} = \{x \in \mathbb{R}^n: x_i \in [l_i, u_i]\}$, where $l, u \in
        \mathbb{R}^n$ and $l_i \leqslant u_i, \forall i \in [n]$.
\end{definition}

In our research we use special kind of \textbf{EO} -- \textit{coordinate-wise
    erroneous oracle}.

\begin{definition}[Erroneous Oracles]
    Let $\varepsilon \in [0, 1)$ be the relative-error, and let
    $x \in \mathcal{C}$. The oracle $\mathcal{O}(\cdot; \varepsilon)$ is called
    an \textit{erroneous oracle}(\textbf{EO}) iff $\forall x \in \mathcal{C}$ it
    returns $\errgrad = \mathcal{O}(x; \varepsilon) \in \mathbb{R}^n$
    satisfying:

    \begin{equation}
        \norm{\errgrad - \nabla f(x)} \leqslant \varepsilon \norm{\nabla f(x)}.
    \end{equation}

    Furthermore, if returned $\errgrad$ satisfies:

    \begin{equation}
        |\errgrad_i - \nabla f(x)_i| \leqslant \varepsilon |\nabla f(x)_i|,
        \exists i \in [n].
    \end{equation}

    Then it is called a \textit{coordinate-wise erroneous oracle \textbf{CWEO}}.

\end{definition}

\section{Computational Experiments}\label{sec:experiments}

We will undergo computational experiments for two scenarios: one is with
the step size, which depends on interation number, the other one -- the step
size depends on Lipschitz-gradient constant.

\subsection{Iteration dependent step size}

In this scenario step size is computed by the following formula:

\begin{equation}
    \eta_t = \min \Bigl\{1, \frac{2}{t + 2}\Bigr\}, \forall t \geqslant 0
\end{equation}

\subsubsection{Case $L_t \leqslant c \cdot L$}

\begin{theorem}[Convergence rate when $L_t \leqslant c \cdot L$]\label{theorem:convergence_rate_Lt_ltq_cL}
    Let $\eta_t$ is a step size at each iteration of the algorithm,
    $L_t \leqslant c \cdot L$, where
    $L$ is Lipschitz-gradient constant, then the following inequality holds:

    \begin{equation}
        f^{t} - f^{*} \leqslant \varepsilon M R + \frac{4 L R^2}{t+2}
    \end{equation}
\end{theorem}

According to the theorem \ref{theorem:convergence_rate_Lt_ltq_cL} follows that
the convergence rate of the algorithm is $O(\frac{1}{t})$.

\begin{theorem}[Stopping criterion when $L_t \leqslant c \cdot L$]
    \label{theorem:stopping_criterion}
    Let $eta_t$ is a step size at each iteration of the algorithm,
    $L_t \leqslant c \cdot L$, where
    $L$ is Lipschitz-gradient constant, then the following inequality holds:

    \begin{equation}
        f^{t + 1} - f^{*} \leqslant \varepsilon M R +
        \frac{4 \max\limits_{t} L_t R^2}{(t+2)^2}
    \end{equation}
\end{theorem}

The implementation of the algorithm is listed below:

\begin{algorithm}[H]\label{alg:aecg_Lt_lqt_cL}
    \SetAlgoLined
    \SetKwInOut{Input}{Input}
    \caption{Adaptive Erroneous Conditional Gradient (AECG) with $L_t
            \leqslant c \cdot L$}
    \Input{$w^0 \in \mathcal{C}, \varepsilon \geqslant 0, c \geqslant 2, L$.}
    set $f^* \leftarrow f(w^0)$\;
    \For{any $t \geqslant 0$}{
    retrieve $\errgrad^t \leftarrow \mathcal{O}(\nabla f(w^t), \varepsilon)$\;
    compute $p^{t+1} \leftarrow LMO(\errgrad^t) - w^t$\;
    compute $\eta_t \leftarrow \min(1, \frac{2}{t+2})$\;
    set $w^{t+1} \leftarrow w^t + \eta_t p^{t+1}$\;
    set $f^* \leftarrow \min(f^*, f^{t+1}))$\;
    set $L_t \leftarrow c \cdot L$\;
    set $M \leftarrow \norm{\nabla f^t}$\;
    set $R \leftarrow \norm{p^{t+1}}$\;
    \If{$f^{t+1} - f^* \leqslant \varepsilon M R + \frac{4 L_t R^2}{(t + 2)^2}$}{
        break\;
    }
    set $t \leftarrow t + 1$\;
    }
\end{algorithm}

where:

\begin{itemize}
    \item $\nabla f(w^t)$ is gradient of $f$ satisfying Lispschitz continuity
          condition:

          \begin{equation}
              \norm{\nabla f(x) - \nabla f(y)}_{2} \leqslant L \norm{x - y}_{2},
              \forall x, y \in \mathcal{C}
          \end{equation}

    \item $\varepsilon$ -- relative error
    \item $\mathcal{O}(\cdot, \cdot)$ -- \textit{Erroneous Oracle (EO)}
    \item $LMO$ -- linear minimization oracle
    \item $p^{t+1}$ -- search direction
    \item $\eta_{t}$ -- chosen step size
    \item $L$ -- Lipschitz-gradient constant
\end{itemize}

The results of the computational experiment are shown in Figure \ref{fig:itereation_dependent_step_size}.

\begin{figure}[h]\label{fig:convergence_rate}
    \begin{center}
        \adjustimage{max size={0.9\linewidth}{0.9\paperheight}}{images/case_Lt_lt_cl.png}
        { \hspace*{\fill} \\}
    \end{center}
    \caption{Computational experiment results, when $L_t \leqslant c \cdot L$}
\end{figure}

\subsubsection{Case $L_t \geqslant L$}

\begin{theorem}[Convergence rate when $L_t \geqslant L$]
    Let $\eta_t$ is a step size at each iteration of the algorithm,
    $L_t \geqslant L$, then the following inequality holds:

    \begin{equation}
        f^{t} - f^{*} \leqslant 2 \varepsilon M R + \frac{4 L R^2}{t+2}
    \end{equation}
\end{theorem}

\begin{theorem}[Stopping criterion when $L_t \geqslant L$]\label{theorem:stopping_criterion}
    Let $eta_t$ is a step size at each iteration of the algorithm,
    $L_t \geqslant L$, where
    $L$ is Lipschitz-gradient constant, then the following inequality holds:

    \begin{equation}
        f^{t + 1} - f^{*} \leqslant \frac{t}{t + 2} +
        \frac{4 \varepsilon M R}{t + 2} +
        \frac{4 \max\limits_{t} L_t R^2}{(t+2)^2}
    \end{equation}
\end{theorem}

The algorithm listing is presented below:

\begin{algorithm}[H]\label{alg:aecg}
    \SetAlgoLined
    \SetKwInOut{Input}{Input}
    \caption{Adaptive Erroneous Conditional Gradient (AECG) with $L_t \geqslant L$}
    \Input{$w^0 \in \mathcal{C}, \varepsilon \geqslant 0$.}
    set $f^* \leftarrow f(w^0)$\;
    set $L_t \leftarrow 0$\;
    \For{any $t \geqslant 0$}{
    retrieve $\errgrad^t \leftarrow \mathcal{O}(\nabla f(w^t), \varepsilon)$\;
    compute $p^{t+1} \leftarrow LMO(\errgrad^t) - w^t$\;
    compute $\eta_t \leftarrow \min(1, \frac{2}{t+2})$\;
    set $w^{t+1} \leftarrow w^t + \eta_t p^{t+1}$\;
    set $f^* \leftarrow \min(f^*, f^{t+1})$\;
    set $M \leftarrow \norm{\nabla f^t}$\;
    set $R \leftarrow \norm{p^{t+1}}$\;
    compute $L_t \leftarrow \max(L_t, \frac{f^{t+1} - f^{t} - \eta_t \vprod{\errgrad, p^{t + 1}}
        - \eta_t \varepsilon M R}{\eta_t^2 \cdot R^2})$\;
    \If{$f^{t+1} - f^* \leqslant \varepsilon M R + \frac{4 L_t R^2}{(t + 2)^2}$}{
        break\;
    }
    set $t \leftarrow t + 1$\;
    }
\end{algorithm}

\begin{figure}[h]\label{fig:convergence_rate_case_Lt_qt_L}
    \begin{center}
        \adjustimage{max size={0.9\linewidth}{0.9\paperheight}}{images/case_Lt_gt_L.png}
        { \hspace*{\fill} \\}
    \end{center}
    \caption{Computational experiment results, when $L_t \geqslant L$}
\end{figure}

\subsection{Lipschitz-constant gradient dependent step size}
\textbf{TBD}

\section{Discussion}\label{sec:discussion}
\textbf{TBD}


\section{Conclusion}\label{sec:conclusion}
\textbf{TBD}

%
% ---- Bibliography ----
%
% BibTeX users should specify bibliography style 'splncs04'.
% References will then be sorted and formatted in the correct style.
%
% \bibliographystyle{splncs04}
% \bibliography{mybibliography}
%
\begin{thebibliography}{8}
    \bibitem{hallak:2024}
    Hallak, N., Kfir Y.: A Study of First-Order Methods with a Deterministic Relative-Error Gradient Oracle. Proceedings of the 41st International
    Conference on Machine Learning \textbf{2}(235), 17313--17332 (2024)
    \url{https://proceedings.mlr.press/v235/hallak24a.html}

    \bibitem{brin:2012}
    Brin, S., Page L.: Reprint of: The anatomy of a large-scale hypertextual
    web search engine, Computer networks, \textbf{2}(56), 3825--3833 (2012)

    \bibitem{anikin:2022}
    Anikin A., Gasnikov A., Gornov A., Kamzolov D., Maximov Y., Nesterov Y.: Efficient numerical methods to solve sparse linear equations
    with application to PageRank. Optimization Methods and Software
    \textbf{2}(37), 907--935, (2022).
    \url{https://doi.org/10.1080/10556788.2020.1858297}

    \bibitem{bomze:2021}
    Bomze, M. I., Rinaldi F., Zeffiro, D.: Frank-Wolfe and friends: a journey
    into projection-free first-order optimization methods, (2021)
    \url{https://arxiv.org/abs/2106.10261}

    \bibitem{stonyakin:2022}
    Stonyakin, F., Kuruzov, I., Polyak, B.: Stopping rules for gradient methods
    for non-convex problems with additive noise in gradient, (2022)
    \url{https://arxiv.org/abs/2205.07544}

    \bibitem{dvurechensky:2015}
    Dvurechensky, P., Gasnikov, A.: Stochastic Intermediate Gradient Method for
    Convex Problems with Inexact Stochastic Oracle.
    arXiv:1411.2876 (2015). \url{https://arxiv.org/abs/1411.2876}

    \bibitem{dvurechensky:2017}
    Dvurechensky, P.: Gradient Method With Inexact Oracle for Composite
    Non-Convex Optimization.
    arXiv:1703.09180 (2017). \url{https://arxiv.org/abs/1703.09180}

    \bibitem{polyak:1987}
    Polyak, B. T.: Introduction to optimization, (1987)

    \bibitem{frankwolfe:1956}
    Frank, M., Wolfe, P.: An algorithm for quadratic programming, Naval research
    logistics quarterly, \textbf{3(1-2)}, 95--110, (1956)

    \bibitem{recht:2019}
    Recht, B., Wright, S.: Optimization for Modern Data Analysis, (2019)
\end{thebibliography}

\clearpage
\section{Proofs for section \ref{sec:experiments}}

For convenience we denote $f(w^t) := f^t$ -- the value of objective
function at the given iteration $t$, $f(w^*) := f^*$ -- minimum value of
objective function.

Let $w^t, w^*, d \in \mathbb{R}, \forall t \geqslant 0$. From the convexity
of $f$, the choice of $p^{t+1}$ and the definition of \textbf{EO}:

\begin{equation}
    -\min_{d \in C}\vprod{\errgrad_t, d - w^t} \geqslant
    \vprod{\errgrad_t, w^t - w^*}
\end{equation}

\begin{equation}\label{eq:sum_of_gradient}
    \vprod{\errgrad^t, w^t - w^*} = \vprod{\errgrad^t - g^t, w^t - w^*} +
    \vprod{g^t, w^t - w^*}
\end{equation}

\begin{equation}\label{eq:grad_to_objective}
    \vprod{g^t, w^t - w^*} \geqslant f^t - f^*
\end{equation}

\begin{equation}\label{eq:errgrad_to_epsilon_grad}
    \norm{\errgrad^t - g^t} \leqslant \varepsilon \norm{g^t}
\end{equation}

According to Cauchy–Bunyakovsky–Schwarz inequality and plugging
\ref{eq:grad_to_objective} and \ref{eq:errgrad_to_epsilon_grad} into
\ref{eq:sum_of_gradient} we have:

\begin{equation}
    \begin{split}
        \vprod{\errgrad^t - g^t, w^t - w^*} +
        \vprod{g^t, w^t - w^*}\geqslant \\
        \\\geqslant- \norm{\errgrad^t - g^t} \norm{w^t - w^*}
        + f^t - f^* \geqslant           \\
        \geqslant - \varepsilon \norm{g^t} R
        + f^t - f^* \geqslant - \varepsilon M R + f^t - f^*
    \end{split}
\end{equation}

Then:

\begin{equation}
    f^t - f^* \leqslant \varepsilon M R -
    \min\limits_{d \in C}\vprod{\errgrad_t, d - w^t}
\end{equation}

\begin{equation}
    f^t - f^* \leqslant \varepsilon M R - \vprod{\errgrad_t, p^{t + 1}}
\end{equation}

From \cite{hallak:2024}:

\begin{equation}\label{eq:descent_property_extracted}
    f^{t + 1} - f^* \leqslant f^t - f^* + \eta_t \vprod{\errgrad, p^{t + 1}}
    + \frac{L_t \eta_t^2}{2}\norm{p^{t + 1}}^2
\end{equation}

\begin{equation}
    \frac{L_t \eta_t^2}{2}\norm{p^{t + 1}}^2 = f^{t + 1} - f^t - \eta_t \vprod{\errgrad_t, p^{t + 1}}
\end{equation}

\begin{equation}
    L_t = \frac{2 (f^{t + 1} - f^t - \eta_t \vprod{\errgrad_t, p^{t + 1}})}
    {\eta_t^2 \norm{p^{t + 1}}^2}
\end{equation}

\begin{proof}[Theorem \ref{theorem:convergence_rate_Lt_ltq_cL}]
    Plugging step size regime into \eqref{eq:descent_property_extracted} yields:

    \begin{equation}\label{eq:basic_proof_of_convergence_rate_Lt_ltq_cL}
        \begin{split}
            f^{t + 1} - f^* \leqslant f^t - f^* + \frac{2}{t + 2}
            \vprod{\errgrad_t, p^{t + 1}} +
            \frac{4 \max\limits_t{L_t} R^2}{(t + 2)^2} \leqslant \\
            \leqslant f^t - f^* + \frac{2}{t + 2} (f^* - f^t + \varepsilon M R) +
            \frac{4 \max\limits_t{L_t} R^2}{(t + 2)^2}           \\
            \leqslant \frac{t}{t + 2}(f^t - f^*) +
            \frac{2 \varepsilon M R}{t + 2} +
            \frac{4 \max\limits_t{L_t} R^2}{(t + 2)^2}
        \end{split}
    \end{equation}

    By induction: for $t = 0$, $f^1 - f^* \leqslant \varepsilon M R +
        \max\limits_t{L_t} R^2$, assume that:

    \begin{equation}\label{eq:convergence_rate_Lt_ltq_cL_proof}
        f^t - f^* \leqslant \varepsilon M R + \frac{\max\limits_t{L_t} R^2}{t + 2}
        \square
    \end{equation}

    We will show that \ref{theorem:convergence_rate_Lt_ltq_cL} holds for $t + 1$.

    Using the assumption \eqref{eq:convergence_rate_Lt_ltq_cL_proof} and relation
    \eqref{eq:basic_proof_of_convergence_rate_Lt_ltq_cL}:

    \begin{equation}
        \begin{split}
            f^{t + 1} - f^* \leqslant \frac{t}{t + 2}(f^t - f^*) +
            \frac{2 \varepsilon M R}{t + 2} +
            \frac{4 \max\limits_t{L_t} R^2}{(t + 2)^2} \leqslant \\
            \leqslant \frac{t}{t + 2}(\varepsilon M R +
            \frac{\max\limits_t{L_t} R^2}{t + 2})
            + \frac{2 \varepsilon M R}{t + 2}
            + \frac{4 \max\limits_t{L_t} R^2}{(t + 2)^2} =       \\
            = \varepsilon M R + \frac{(t + 1)}{(t + 2)^2} \max\limits_t{L_t} R^2
            \square
        \end{split}
    \end{equation}


\end{proof}


\end{document}
